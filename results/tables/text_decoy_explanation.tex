\begin{verbatim}
Be careful not to fall for the Decoy Effect and the Phantom Decoy Effect when evaluating candidates.

### Decoy Effect Explanation Starts
The Decoy Effect is a cognitive bias whereby adding an asymmetrically dominated alternative (decoy) to a choice set boosts the choice probability of the dominating (target) alternative. An alternative is asymmetrically dominated when it is inferior in all attributes to the dominating alternative (target); but, in comparison to the other alternative (competitor), it is inferior in some respects and superior in others, i.e., it is only partially dominated.

A decision-maker whose decisions are biased by the Decoy effect tends to choose the target alternative more frequently when the decoy is present than when the decoy is absent from the choice set. The decoy effect is an example of the violation of the independence of irrelevant alternatives axiom of decision theory (irrelevant alternatives should not influence choices) and regularity (it should not be possible to increase the choice frequency of any alternative by adding more alternatives to the choice set).

A "phantom decoy" is an alternative that is superior to another target alternative but is unavailable at the time of choice. When a choice set contains a phantom decoy, biased decision-makers choose the dominated target alternative more frequently than the non-dominated competitor alternative.

Here is an example of the Decoy Effect. Suppose there is a job ad for an interpreter with German and English. Knowledge of each of the two languages is equally important. Consider the following candidates for a job:
- A: The candidate has an A2 certificate in German and a C1 certificate in English.
- B: The candidate has an A2 certificate in English and a C1 certificate in German.
- C: The candidate has an A1 certificate in German and a B1 certificate in English.

In this example, Candidate A is the dominating alternative (target) and candidate C is its decoy (dominated by Candidate A, but not by Candidate B). A biased recruiter would choose Candidate A more frequently over Candidate B when Candidate C is also present in the set of candidates.

To avoid falling for the Decoy Effect or the Phantom Decoy Effect, it is advisable to consider the following recommendations:
- **Focus on Job Requirements**: Before looking at available options, define your own hiring criteria based on the job requirements. Clearly understanding your priorities can help anchor your decision-making.
- **Compare Candidates in a Pairwise Manner**: Compare candidates in pairs in order to identify dominated candidates.
- **Ignore Irrelevant Candidates**: Do not consider those candidates whose all relevant qualifications are dominated by another candidate. Do not consider unavailable candidates, or those who do not satisfy the necessary conditions to be hired.
- **Take Your Time**: Don't make impulsive decisions. Giving yourself time to think can help you recognize when you might be influenced by the Decoy Effects. 

By following these steps, you can reduce the impact of the Decoy Effect and make more rational, well-informed decisions that truly reflect hiring needs.

### Decoy Effect Explanation Ends
\end{verbatim}
